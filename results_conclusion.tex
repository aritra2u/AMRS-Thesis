\chapter{Results}
	\section{Tests}
		Tables \ref{table:test_results_3en}, \ref{table:test_results_rj}, \ref{table:test_results_eartle}, \ref{table:test_results_franhale}, \ref{table:test_results_massdosage} illustrates the tests have been performed on \(5\) Last.FM users' histories (``3en'', ``RJ'', ``eartle'', ``franhale'', ``massdosage'') by varying the parameters such as the number of similar users selected for collaborative filtering, the number of songs to define a mood, and the set of weightages (artist, loudness, tempo) given to each property of the song to calculate its similarity with others.

		The confidence is the measure of similarity of a ``mood window'', found in the history of another user having a very similar musical taste, which is most ``similar'' to that of the ``mood window'' considered for the current user.
		
		The most recent \(t\) songs are used for testing purposes and the recommendations are based on the songs following the first \(t\). The rank of a song in the recommendation that occurs in the set of test songs defined above is also noted down.
		
\begin{table}[h!]
\centering
\begin{tabular}{ | c | c | c || c | c | }
\hline
Similar Users	& Mood Length	& Weights							&Confidence	&Rank\\
\hline \hline
50			& 5			& \(\rfrac{1}{3}, \rfrac{1}{3}, \rfrac{1}{3}\)	&48.45 \%		&3718\\
\hline
75			& 5			& \(\rfrac{1}{3}, \rfrac{1}{3}, \rfrac{1}{3}\)	&48.45 \%		&3879\\
\hline
100			& 5			& \(\rfrac{1}{3}, \rfrac{1}{3}, \rfrac{1}{3}\)	&48.45 \%		&84\\
\hline
150			& 5			& \(\rfrac{1}{3}, \rfrac{1}{3}, \rfrac{1}{3}\)	&51.01 \%		&135\\
\hline
200			& 5			& \(\rfrac{1}{3}, \rfrac{1}{3}, \rfrac{1}{3}\)	&52.63 \%		&211\\
\hline
50			& 10			& \(\rfrac{1}{3}, \rfrac{1}{3}, \rfrac{1}{3}\)	&45.06 \%		&3418\\
\hline
75			& 10			& \(\rfrac{1}{3}, \rfrac{1}{3}, \rfrac{1}{3}\)	&45.50 \%		&4751\\
\hline
100			& 10			& \(\rfrac{1}{3}, \rfrac{1}{3}, \rfrac{1}{3}\)	&46.93 \%		&1722\\
\hline
50			& 5			& \(\rfrac{1}{5}, \rfrac{2}{5}, \rfrac{2}{5}\)	&43.28 \%		&4033\\
\hline
50			& 5			& \(\rfrac{3}{5}, \rfrac{1}{5}, \rfrac{1}{5}\)	&60.53 \%		&3632\\
\hline
75			& 5			& \(\rfrac{3}{5}, \rfrac{1}{5}, \rfrac{1}{5}\)	&60.03 \%		&4367\\
\hline
100			& 5			& \(\rfrac{3}{5}, \rfrac{1}{5}, \rfrac{1}{5}\)	&62.10 \%		&78\\
\hline
150			& 5			& \(\rfrac{3}{5}, \rfrac{1}{5}, \rfrac{1}{5}\)	&64.73 \%		&120\\
\hline
\end{tabular}
\caption{Test Results for Last.FM user: \emph{3en}}
\label{table:test_results_3en}
\end{table}

\begin{table}[h!]
\centering
\begin{tabular}{ | c | c | c || c | c | }
\hline
Similar Users	& Mood Length	& Weights							&Confidence	&Rank\\
\hline \hline
50			& 5			& \(\rfrac{1}{3}, \rfrac{1}{3}, \rfrac{1}{3}\)	&50.43 \%		&N/A\\
\hline
75			& 5			& \(\rfrac{1}{3}, \rfrac{1}{3}, \rfrac{1}{3}\)	&50.43 \%		&N/A\\
\hline
100			& 5			& \(\rfrac{1}{3}, \rfrac{1}{3}, \rfrac{1}{3}\)	&50.43 \%		&7608\\
\hline
150			& 5			& \(\rfrac{1}{3}, \rfrac{1}{3}, \rfrac{1}{3}\)	&50.43 \%		&8828\\
\hline
200			& 5			& \(\rfrac{1}{3}, \rfrac{1}{3}, \rfrac{1}{3}\)	&52.40 \%		&10018\\
\hline
50			& 10			& \(\rfrac{1}{3}, \rfrac{1}{3}, \rfrac{1}{3}\)	&48.91 \%		&N/A\\
\hline
75			& 10			& \(\rfrac{1}{3}, \rfrac{1}{3}, \rfrac{1}{3}\)	&48.91 \%		&N/A\\
\hline
100			& 10			& \(\rfrac{1}{3}, \rfrac{1}{3}, \rfrac{1}{3}\)	&48.91 \%		&N/A\\
\hline
50			& 5			& \(\rfrac{1}{5}, \rfrac{2}{5}, \rfrac{2}{5}\)	&43.28 \%		&N/A\\
\hline
50			& 5			& \(\rfrac{3}{5}, \rfrac{1}{5}, \rfrac{1}{5}\)	&66.15 \%		&N/A\\
\hline
75			& 5			& \(\rfrac{3}{5}, \rfrac{1}{5}, \rfrac{1}{5}\)	&66.15 \%		&N/A\\
\hline
100			& 5			& \(\rfrac{3}{5}, \rfrac{1}{5}, \rfrac{1}{5}\)	&66.15 \%		&7095\\
\hline
150			& 5			& \(\rfrac{3}{5}, \rfrac{1}{5}, \rfrac{1}{5}\)	&66.15 \%		&7767\\
\hline
\end{tabular}
\caption{Test Results for Last.FM user: \emph{RJ}}
\label{table:test_results_rj}
\end{table}

\begin{table}[h!]
\centering
\begin{tabular}{ | c | c | c || c | c | }
\hline
Similar Users	& Mood Length	& Weights							&Confidence	&Rank\\
\hline \hline
50			& 5			& \(\rfrac{1}{3}, \rfrac{1}{3}, \rfrac{1}{3}\)	&50.43 \%		&400\\
\hline
75			& 5			& \(\rfrac{1}{3}, \rfrac{1}{3}, \rfrac{1}{3}\)	&50.43 \%		&607\\
\hline
100			& 5			& \(\rfrac{1}{3}, \rfrac{1}{3}, \rfrac{1}{3}\)	&48.37 \%		&736\\
\hline
150			& 5			& \(\rfrac{1}{3}, \rfrac{1}{3}, \rfrac{1}{3}\)	&51.94 \%		&1095\\
\hline
200			& 5			& \(\rfrac{1}{3}, \rfrac{1}{3}, \rfrac{1}{3}\)	&51.94 \%		&1428\\
\hline
50			& 10			& \(\rfrac{1}{3}, \rfrac{1}{3}, \rfrac{1}{3}\)	&48.91 \%		&2632\\
\hline
75			& 10			& \(\rfrac{1}{3}, \rfrac{1}{3}, \rfrac{1}{3}\)	&48.91 \%		&3736\\
\hline
100			& 10			& \(\rfrac{1}{3}, \rfrac{1}{3}, \rfrac{1}{3}\)	&46.91 \%		&4304\\
\hline
50			& 5			& \(\rfrac{1}{5}, \rfrac{2}{5}, \rfrac{2}{5}\)	&43.28 \%		&563\\
\hline
50			& 5			& \(\rfrac{3}{5}, \rfrac{1}{5}, \rfrac{1}{5}\)	&66.15 \%		&363\\
\hline
75			& 5			& \(\rfrac{3}{5}, \rfrac{1}{5}, \rfrac{1}{5}\)	&66.15 \%		&555\\
\hline
100			& 5			& \(\rfrac{3}{5}, \rfrac{1}{5}, \rfrac{1}{5}\)	&63.88 \%		&650\\
\hline
150			& 5			& \(\rfrac{3}{5}, \rfrac{1}{5}, \rfrac{1}{5}\)	&64.59 \%		&970\\
\hline
\end{tabular}
\caption{Test Results for Last.FM user: \emph{eartle}}
\label{table:test_results_eartle}
\end{table}

\begin{table}[h!]
\centering
\begin{tabular}{ | c | c | c || c | c | }
\hline
Similar Users	& Mood Length	& Weights							&Confidence	&Rank\\
\hline \hline
50			& 5			& \(\rfrac{1}{3}, \rfrac{1}{3}, \rfrac{1}{3}\)	&49.71 \%		&4744\\
\hline
75			& 5			& \(\rfrac{1}{3}, \rfrac{1}{3}, \rfrac{1}{3}\)	&49.83 \%		&629\\
\hline
100			& 5			& \(\rfrac{1}{3}, \rfrac{1}{3}, \rfrac{1}{3}\)	&49.71 \%		&674\\
\hline
150			& 5			& \(\rfrac{1}{3}, \rfrac{1}{3}, \rfrac{1}{3}\)	&51.10 \%		&4351\\
\hline
200			& 5			& \(\rfrac{1}{3}, \rfrac{1}{3}, \rfrac{1}{3}\)	&51.48 \%		&4363\\
\hline
50			& 10			& \(\rfrac{1}{3}, \rfrac{1}{3}, \rfrac{1}{3}\)	&47.49 \%		&3160\\
\hline
75			& 10			& \(\rfrac{1}{3}, \rfrac{1}{3}, \rfrac{1}{3}\)	&47.49 \%		&3135\\
\hline
100			& 10			& \(\rfrac{1}{3}, \rfrac{1}{3}, \rfrac{1}{3}\)	&47.54 \%		&3225\\
\hline
50			& 5			& \(\rfrac{1}{5}, \rfrac{2}{5}, \rfrac{2}{5}\)	&43.28 \%		&4422\\
\hline
50			& 5			& \(\rfrac{3}{5}, \rfrac{1}{5}, \rfrac{1}{5}\)	&64.43 \%		&4589\\
\hline
75			& 5			& \(\rfrac{3}{5}, \rfrac{1}{5}, \rfrac{1}{5}\)	&66.18 \%		&470\\
\hline
100			& 5			& \(\rfrac{3}{5}, \rfrac{1}{5}, \rfrac{1}{5}\)	&64.43 \%		&471\\
\hline
150			& 5			& \(\rfrac{3}{5}, \rfrac{1}{5}, \rfrac{1}{5}\)	&64.43 \%		&5227\\
\hline
\end{tabular}
\caption{Test Results for Last.FM user: \emph{franhale}}
\label{table:test_results_franhale}
\end{table}

\begin{table}[h!]
\centering
\begin{tabular}{ | c | c | c || c | c | }
\hline
Similar Users	& Mood Length	& Weights							&Confidence	&Rank\\
\hline \hline
50			& 5			& \(\rfrac{1}{3}, \rfrac{1}{3}, \rfrac{1}{3}\)	&48.89 \%		&N/A\\
\hline
75			& 5			& \(\rfrac{1}{3}, \rfrac{1}{3}, \rfrac{1}{3}\)	&50.09 \%		&9857\\
\hline
100			& 5			& \(\rfrac{1}{3}, \rfrac{1}{3}, \rfrac{1}{3}\)	&50.09 \%		&11647\\
\hline
150			& 5			& \(\rfrac{1}{3}, \rfrac{1}{3}, \rfrac{1}{3}\)	&51.10 \%		&6587\\
\hline
200			& 5			& \(\rfrac{1}{3}, \rfrac{1}{3}, \rfrac{1}{3}\)	&51.48 \%		&8008\\
\hline
50			& 10			& \(\rfrac{1}{3}, \rfrac{1}{3}, \rfrac{1}{3}\)	&48.08 \%		&N/A\\
\hline
75			& 10			& \(\rfrac{1}{3}, \rfrac{1}{3}, \rfrac{1}{3}\)	&48.08 \%		&2584\\
\hline
100			& 10			& \(\rfrac{1}{3}, \rfrac{1}{3}, \rfrac{1}{3}\)	&48.08 \%		&2887\\
\hline
50			& 5			& \(\rfrac{1}{5}, \rfrac{2}{5}, \rfrac{2}{5}\)	&43.28 \%		&N/A\\
\hline
50			& 5			& \(\rfrac{3}{5}, \rfrac{1}{5}, \rfrac{1}{5}\)	&63.29 \%		&N/A\\
\hline
75			& 5			& \(\rfrac{3}{5}, \rfrac{1}{5}, \rfrac{1}{5}\)	&65.97 \%		&7819\\
\hline
100			& 5			& \(\rfrac{3}{5}, \rfrac{1}{5}, \rfrac{1}{5}\)	&65.97 \%		&9345\\
\hline
150			& 5			& \(\rfrac{3}{5}, \rfrac{1}{5}, \rfrac{1}{5}\)	&68.19 \%		&5749\\
\hline
\end{tabular}
\caption{Test Results for Last.FM user: \emph{massdosage}}
\label{table:test_results_massdosage}
\end{table}

	\section{Inference}
		The higher the ``confidence'' value, the more similar is its corresponding ``mood'' to that of the current user's present mood and thus the chances of the current user liking the song heard by the other user right after the above window get higher.
		
		The rank illustrates the success of the recommendation by cross-checking with the songs the current user has actually heard next, i.e, the set of test songs chosen above. This however, determines the success of the tool in the current domain of the \(999,999\) songs.
		
		Based on the test results, it can inferred that a \emph{mood length} of \(5\) based on the histories of \(75\) other users with similar musical taste gives a set of good results. The song's property weightage of \(\rfrac{3}{5}, \rfrac{1}{5}, \rfrac{1}{5}\) is also seen to contribute well to the recommendations. Hence, these values are taken to be the default values of the tool. These can however be customised to give more weightage to a parameter of choice and thus get different sets of results.
	
	\section{Improvements}
		This recommendation system has been built more as a \emph{proof of concept} than a competitive tool. A lot of things may enhance the performance the results of the recommendation and to make it even more accurate. Some of them have been discussed below.

 \begin{itemize}
 	\item Million Song Dataset: The global dictionary of songs has been restricted to 1,000,000 and recommendations are made from this dataset only. Adding more songs, will improve the quality of recommendations.
 	\item User History: There is a certain amount of loss while fetching the user history and cleaning it up. It corrupts the determined ``mood'' of the user. Using a cleaner method and a reliable source of history should be able to tackle this matter.
 	\item Users: Currently, the set of users being used for collaborative filtering is not only limited but also very random and does not encompass the different moods and interests of music listeners.
 	\item MFCC: There are several other parameters as obtained from the MFCC analysis of a song. These included appropriately in the computation would significantly bring out better results.
 	\item Feedback: If a user skips a certain song, the recommendation engine should learn not to present her with the song anytime soon again. The weights for each parameters can be dynamically changed as per the user feedback using machine learning techniques.
 \end{itemize} 
	
	\section{Summary}
\begin{itemize}
	\item Mood: This plays a very significant role in the process of recommendation and has not been very widely tapped. The type of songs a user has been listening to tells a great deal about her current mood and proves to be prominent factor in determining the songs she might want to hear next.
	\item Popularity: This parameter has been gaining importance as new artists come up with various types of songs, which are hard to categorize and thus to be recommended. Including popularity gives a way to recommend any trending song even if it does not match the user's preferences.
\end{itemize}